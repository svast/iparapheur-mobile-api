\section{Dossiers}

\subsection{getDossier}
\shadowbox{\verb|/parapheur/api/getDossier|
}\\


Le flag \verb|recuperable| indique si le dernier acteur qui a effectué une action de validation récupérable (visa) peut récupérer le dossier.

Le champ \verb|nomTdt| correspond au nom du tiers de télétransmission renseigné dans le sous type associé au dossier (FAST, S2LOW ou SRCI).

Le champ \verb|protocoleTdt| correspond au protocole du tiers de télétransmission sélectionné pour ce dossier (HELIOS ou ACTES).

Le champ \verb|type| correspond au type métier du dossier.

Le champ \verb|sousType| correspond au sous type métier du dossier.

Le champ \verb|lectureObligatoire| signifie que la lecture du document principal est obligatoire à l'étape de signature.

Le champ \verb|signatureNumeriqueObligatoire| signifie que la signature électronique est obligatoire (pas de signature "papier")

Le champ \verb|actions| donne les possibilités d'actions pour l'utilisateur

Le champ \verb|metadonnees| renseigne les métadonnées du circuit et leurs valeurs pour le dossier actuel.

La partie documents de la réponse à \verb|getDossier| contient un champ \verb|downloadUrl| (pour avoir le chemin complet on effectue la concaténation du préfixe et de l'url\footnote{Dans le cas présent ça donne: http://mon.parapheur.fr/alfresco/service/api/node/workspace/SpacesStore/1b4b7715-c053-4cea-bed7-5be34373d565/content?alf\_ticket=TICKET\_6a04816348fe8d9aab05ec7a43cfed22bdcdedd2}) un champ \verb|size| en octets qui représente la taille du document éventuellement un champ \verb|visuelPdfUrl| qui renvoie une version pdf du document (voir extrait~\ref{snip:getDossier_out}). Un dossier contient un document principal (le premier de la liste) et éventuellement des annexes.

\begin{codesnippet}
\inputminted[frame=single,linenos,fontsize=\footnotesize]{javascript}{extraits/getDossier_in.js}
\caption{getDossier in}
\label{snip:getDossier_out}
\end{codesnippet}


\subsection{getImages}
\shadowbox{\verb|/parapheur/api/getImages|
}\\

Permet de récupérer le visuel du document principal sous forme de liste d'images (voir extrait~\ref{snip:getImages_out}).

\begin{codesnippet}
\inputminted[frame=single,linenos,fontsize=\footnotesize]{javascript}{extraits/getImages_in.js}
\caption{getImages in}
\label{snip:getImages_in}
\end{codesnippet}


\subsection{getMetadonnees}
\shadowbox{\verb|/parapheur/api/getMetadonnees|
}\\

Permet de récupérer les métadonnées relatives aux type et sous-typé donnés (voir extrait~\ref{snip:getMetadonnees_out}).

\begin{codesnippet}
\inputminted[frame=single,linenos,fontsize=\footnotesize]{javascript}{extraits/getMetadonnees_in.js}
\caption{getMetadonnees in}
\label{snip:getMetadonnees_in}
\end{codesnippet}


\subsection{deleteNodes}
\shadowbox{\verb|/parapheur/api/deleteNodes|
}\\

Supprime les noeuds, fournis sous forme de liste.

\begin{codesnippet}
\inputminted[frame=single,linenos,fontsize=\footnotesize]{javascript}{extraits/deleteNodes_in.js}
\caption{deleteNodes in}
\label{snip:deleteNodes_in}
\end{codesnippet}


\subsection{Actions}

Toutes les actions retournent des codes d'erreur en cas d'échec. Le \verb|nom| du dossier suivi de \verb|success| en cas de réussite, ou d'un code d'erreur en cas d'échec.
Toutes les fonctions suivantes peuvent renvoyer l'erreur \verb|app.ajax.msg.autorisation| si l'utilisateur n'a pas l'autorisation de faire cette action.

\subsubsection{visa}
\shadowbox{\verb|/parapheur/api/visa|
}\\

Permet de viser/émettre un dossier. Voir ~\ref{table:possible_error} pour les codes d'erreurs retournés.

\begin{figure}
	\begin{center}
        \begin{tabular}{c|c|c}
	        \hline
	        Message retourné & Erreur \\
	        \hline
	        app.ajax.msg.nameerror & Un dossier du même nom existe déjà \\
	        app.ajax.msg.circuit & Le circuit n'est pas défini\\
	        app.ajax.msg.metadata & Toutes les métadonnées obligatoire ne sont pas renseignées\\
	        app.ajax.msg.type & Type/Sous-Type non défini(s)\\
	        app.ajax.msg.document & Aucun document n'est défini pour ce dossier\\
        \end{tabular}
    \end{center}
    \caption{Erreurs possibles}
    \label{table:possible_error}
\end{figure}

\begin{codesnippet}
\inputminted[frame=single,linenos,fontsize=\footnotesize]{javascript}{extraits/visa_in.js}
\caption{visa in}
\label{snip:visa_in}
\end{codesnippet}


\subsubsection{reject}
\shadowbox{\verb|/parapheur/api/reject|
}\\

Rejet d'un dossier.

\begin{codesnippet}
\inputminted[frame=single,linenos,fontsize=\footnotesize]{javascript}{extraits/reject_in.js}
\caption{reject in}
\label{snip:reject_in}
\end{codesnippet}


\subsubsection{raz}
\shadowbox{\verb|/parapheur/api/razDossier|
}\\

Remise à zéro d'un dossier rejeté.

\begin{codesnippet}
\inputminted[frame=single,linenos,fontsize=\footnotesize]{javascript}{extraits/razDossier_in.js}
\caption{razDossier in}
\label{snip:razDossier_in}
\end{codesnippet}


\subsubsection{remorse}
\shadowbox{\verb|/parapheur/api/remorseDossier|
}\\

Récupération d'un dossier émis.

\begin{codesnippet}
\inputminted[frame=single,linenos,fontsize=\footnotesize]{javascript}{extraits/remorseDossier_in.js}
\caption{remorseDossier in}
\label{snip:remorseDossier_in}
\end{codesnippet}

\subsubsection{archive}
\shadowbox{\verb|/parapheur/api/archive|
}\\

Archivage d'un dossier terminé.

\begin{codesnippet}
\inputminted[frame=single,linenos,fontsize=\footnotesize]{javascript}{extraits/archiveDossier_in.js}
\caption{archiveDossier in}
\label{snip:archiveDossier_in}
\end{codesnippet}

\subsection{Création}

\subsubsection{beginCreateDossier}
\shadowbox{\verb|/parapheur/api/createDossier|
}\\

Crée un dossier vide avec un nom temporaire sous la forme \verb|Sans Titre| suivi du premier chiffre disponible.

\begin{codesnippet}
\inputminted[frame=single,linenos,fontsize=\footnotesize]{javascript}{extraits/createDossier_in.js}
\caption{createDossier requête entrante}
\label{snip:getDossier_in}
\end{codesnippet}

\begin{codesnippet}
\inputminted[frame=single,linenos,fontsize=\footnotesize]{javascript}{extraits/createDossier_out.js}
\caption{createDossier requête sortante}
\label{snip:getDossier_out}
\end{codesnippet}


\subsubsection{setCircuit}
\shadowbox{\verb|/parapheur/api/setCircuit|
}\\

Défini un circuit pour le dossier donné.

\begin{codesnippet}
\inputminted[frame=single,linenos,fontsize=\footnotesize]{javascript}{extraits/setCircuit_in.js}
\caption{setCircuit requête entrante}
\label{snip:setCircuit_in}
\end{codesnippet}



\subsubsection{setDossierProperties}
\shadowbox{\verb|/parapheur/api/setDossierProperties|
}\\

Enregistre les propriétés sur le dossier donné.

\begin{codesnippet}
\inputminted[frame=single,linenos,fontsize=\footnotesize]{javascript}{extraits/setDossierProperties_in.js}
\caption{setDossierProperties requête entrante}
\label{snip:setDossierProperties_in}
\end{codesnippet}


\subsubsection{addDocument}
\shadowbox{\verb|/parapheur/api/addDocument|
}\\

Ajoute un document au dossier donné. Retourne \verb|success| suivi de la référence du document créé (voir extrait~\ref{snip:addDocument_visuel_in}). En cas d'erreur, retourne \verb|error| suivi du type d'erreur rencontré.

\begin{codesnippet}
\inputminted[frame=single,linenos,fontsize=\footnotesize]{javascript}{extraits/addDocument_visuel_out.js}
\caption{addDocument requête sortante}
\label{snip:addDocument_visuel_out}
\end{codesnippet}


\subsubsection{removeDocument}
\shadowbox{\verb|/parapheur/api/removeDocument|
}\\

Supprime un document. Retourne le résultat de l'opération.

\begin{codesnippet}
\inputminted[frame=single,linenos,fontsize=\footnotesize]{javascript}{extraits/removeDocument_in.js}
\caption{removeDocument requête entrante}
\label{snip:removeDocument_in}
\end{codesnippet}

\begin{codesnippet}
\inputminted[frame=single,linenos,fontsize=\footnotesize]{javascript}{extraits/removeDocument_out.js}
\caption{removeDocument requête sortante}
\label{snip:removeDocument_out}
\end{codesnippet}


\subsubsection{finalizeCreateDossier}
\shadowbox{\verb|/parapheur/api/finalizeCreateDossier|
}\\

Permet d'autoriser ou de refuser l'émission du dossier donné.

\begin{codesnippet}
\inputminted[frame=single,linenos,fontsize=\footnotesize]{javascript}{extraits/finalizeCreateDossier_in.js}
\caption{finalizeCreateDossier requête entrante}
\label{snip:finalizeCreateDossier_in}
\end{codesnippet}

\subsection{Archives}

\subsubsection{getArchives}
\shadowbox{\verb|/parapheur/api/getArchives|
}\\

Retourne une liste d'entêtes d'archives. (voir extraits~\ref{snip:getArchives_in} et~\ref{snip:getArchives_out}).

Lorsque l'on limite le nombre ($n$) de résultats \verb|getArchives| retourne au plus $n + 1$ résultats afin d'indiquer une page suivante potentielle. Si \verb|pageSize| vaut 0 alors tout les résultats sont retournés. Voir ~\ref{Filtres} pour les filtres, le fonctionnement général étant le même.

\begin{codesnippet}
\inputminted[frame=single,linenos,fontsize=\footnotesize]{javascript}{extraits/getArchives_in.js}
\caption{getArchives requête entrante}
\label{snip:getArchives_in}
\end{codesnippet}

\subsubsection{getProp}
\shadowbox{\verb|/parapheur/api/getProp|
}\\

Permet de télécharger la propriété de type content donnée.\\
Attention : Cette fonction s'appelle en \verb|GET|.\\

Exemple d'utilisation :\\
Récupération du document original d'une archive :\\
\verb|parapheur/api/getProp?dossier=workspace://SpacesStore/47d8ab48-3704-41b3-aed7-4ee1ee0a6119&property=ph:original|

