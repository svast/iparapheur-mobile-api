% rubber: setlist arguments --shell-escape

\documentclass[a4paper]{article}
%, twoside
\usepackage[francais]{babel}
\usepackage[utf8]{inputenc}
\usepackage{amsmath}

\usepackage[T1]{fontenc}
\usepackage{lmodern}

%%% Use verdana :X
% winfonts hack ... don't wanna do it.
%\fontfamily{verdana}\selectfont


%\usepackage[parfill]{parskip}
%\usepackage{fourier}
\usepackage{algorithmic, algorithm}
\floatname{algorithm}{Algorithme} 
%%% for \url
\usepackage{xcolor}% http://ctan.org/pkg/xcolor
\usepackage{hyperref}% http://ctan.org/pkg/hyperref
\hypersetup{
  colorlinks=true,
  linkcolor=black,
  urlcolor=black!70!black
}% define a command to input emails
\newcommand{\email}[1]{\href{mailto:#1}{\nolinkurl{#1}}}

\usepackage{setspace}
\usepackage[head=20pt,headsep=1cm,margin=1.5cm,driver=xetex,includeheadfoot]{geometry}
%\addtolength{\topmargin}{0.5cm}
\usepackage{fancyhdr}
\usepackage{fancybox}
\usepackage{pgf}
%\usepackage{wrapfig}
\usepackage{amsmath}
\usepackage{fancyvrb}
\usepackage{color}

\usepackage{colortbl}
\usepackage{minted}
\usemintedstyle{autumn}
%\usepackage{tabularx}
%\pagestyle{fancy}
%\usepackage[varg]{pxfonts} 
\setcounter{secnumdepth}{4}% 
\setcounter{tocdepth}{4}%


%%%%% pour \bigsqcap(sans avoir à importer tout pxfonts %%%%%
\DeclareSymbolFont{largesymbolsA}{U}{pxexa}{m}{n}
\SetSymbolFont{largesymbolsA}{bold}{U}{pxexa}{bx}{n}
\DeclareFontSubstitution{U}{pxexa}{m}{n}
\def\re@DeclareMathSymbol#1#2#3#4{%
    \let#1=\undefined
    \DeclareMathSymbol{#1}{#2}{#3}{#4}}

\re@DeclareMathSymbol{\bigsqcap}{\mathop}{largesymbolsA}{6}
%%%%% \bigsqcap %%%%%

%%%%% definis le RETURN qui n'est pas définis dans algorithm
\def\RETURN{\STATE \textbf{return}~}
%%%%

%%%% defs 
%\renewcommand{\chaptermark}[1]{\markright{\thechapter\ #1}}
%\renewcommand{\sectionmark}[1]{\markright{\thesection\ #1}}
%\fancyhead[R]{\thepage}
%\fancyhead[C]{\scshape\rightmark}
%\fancyhead[L]{}
%\fancyfoot[C]{}
\renewcommand{\headrulewidth}{0pt}
%%%%


%%%% Macros persos
\newenvironment{divs}[1][10cm]
  {\begin{minipage}{#1}
   \centering}
  {\end{minipage}}

\newenvironment{doublespacediv}[1][10cm]
  {\begin{divs}[#1]
   \begin{doublespace}}
  {\end{doublespace}
   \end{divs}}


\newenvironment{topsty}[1][10cm]
  {\begin{divs}[#1]
   \begin{large}}
  {\end{large}
   \end{divs}}

%\newcommand{\memtitle}[1][]{
%  \begin{doublespacediv}[#1]
%    \Huge \textbf
%    { #2}
%    
%   
%    \normalsize
%  \end{doublespacediv}
%}

% minted stuff
\renewcommand{\theFancyVerbLine}{\sffamily
\textcolor[rgb]{0.5,0.5,1.0}{\scriptsize
\oldstylenums{\arabic{FancyVerbLine}}}}

%\newcommand{\inputmintedprops}[1]{
%	\inputminted[frame=single,linenos,fontsize=\footnotesize]{properties}{#1}
%}	

%\newcommand{\inputmintedscript}[1]{
%	\inputminted[frame=single,linenos,fontsize=\footnotesize]{bash}{#1}
%}

%%%%

\floatstyle{plain}
\newfloat{codesnippet}{thp}{lop}
\floatname{codesnippet}{Extrait}

\renewcommand{\textfraction}{0.05}
\renewcommand{\topfraction}{0.8}
\renewcommand{\bottomfraction}{0.8}
\renewcommand{\floatpagefraction}{0.75}


\newtheorem{definition}{Définition} % les définitions et les théorèmes sont
\newtheorem{theorem}{Théorème}[section]    % numérotés par section
\newtheorem{proposition}[theorem]{Proposition}
\newtheorem{example}{Exemple}



%%% entêtes et pieds de pages
\pagestyle{fancy}
\renewcommand\headrulewidth{0.4pt}
\fancyhead{}
\fancyhead[C]{
\begin{tabular}{|c|c|c|c|}
	\hline
	\cellcolor[gray]{.9}Projet & \pgfimage[height=7mm]{images/parapheur} & \cellcolor[gray]{.9}Rédacteur & Emmanuel Peralta \\
	\cellcolor[gray]{.9}       & \href{http://paraphelec.adullact.net}{http://paraphelec.adullact.net} &\cellcolor[gray]{.9} & \email{e.peralta@adullact-projet.coop} \\
	\hline
\end{tabular}
\vspace{1mm}
}

\renewcommand\footrulewidth{0.4pt}
\fancyfoot[C]{}
\fancyfoot[R]{\thepage}
\fancyfoot[L]{\textbf{API Mobile *DRAFT*}\\ 
\today}

\begin{document}
	\thispagestyle{fancy}
\begin{titlepage}
\begin{center}
\begin{topsty}[14cm]
	\hbox{
	
	\begin{minipage}{0.5\textwidth}
	\begin{flushleft} 
			\pgfimage[height=1.5cm]{images/parapheur}
	\end{flushleft}
	\end{minipage}

	\begin{minipage}{0.5\textwidth}
	\begin{flushright} 
			\pgfimage[height=2cm]{images/adullact}
	\end{flushright}
	\end{minipage}
	}

\hrule


%\Large
%\textsc{Académie de Montpellier}\\
%\Huge
%\textsc{Université Montpellier II}\\
%\Large
%\begin{doublespace}
%\textsc{-~Sciences et Techniques du Languedoc~-}
%\end{doublespace}
\end{topsty}

\vfill

\begin{doublespacediv}[14cm]
  \Huge \textbf
  {Projet i-Parapheur v3.4.0}\\
	\vspace{5mm}
	\Large
  Documentation API mobile
  \normalsize
\end{doublespacediv}




\vspace{7mm}

%\large{Spécialité} : \textbf{Professionnelle et Recherche unifiée en Informatique}\\

\vfill

%\begin{doublespacediv}[14cm]
%\centering
%\Huge
%\textbf{Apprentissage Supervisé en Robotique}\\
%\Large
%par\\
%\textbf{Emmanuel PERALTA}
%\end{doublespacediv}

%\vfill
%\begin{divs}
%Date de soutenance : \textbf{30 Juin 2009}\\
%	Sous la direction de \textbf{Éric BOURREAU}
%\end{divs}
%>{\columncolor[gray]{0.9}}l
\begin{tabular}{ccccc}
	\textbf{Projet} & \textbf{Redacteur} & \textbf{Date} & \textbf{Statut} & \textbf{Diffusé le} \\
	\hline
	i-Parapheur & Emmanuel Peralta & 24/05/12 & Rédaction & \today \\
	%\hline
	%Projet & i-Parapheur & Rédacteur & Emmanuel Peralta \\
	%\hline
	%Objet & \multicolumn{3}{c|}{ Paramétrage LDAP} \\
	%\hline
	%Date & \date & Statut & En cours de rédaction & Diffusé le -  \\
	%\hline
\end{tabular}
\vfill
\begin{tabular}{cccc}
	%\hline
	%\rowcolor[gray]{.9} 
	\textbf{Organisme} & \textbf{Nom} & \textbf{Fonction} & \textbf{Validé le}\\
	\hline
	ADULLACT & Pascal KUCZYNSKI & Directeur Technique & - \\
	%\hline
	ADULLACT-Projet & Stéphane VAST & Ingénieur Fonctionnel & -\\
	%\hline
	SECLAB & Paul Merlin & Software Engineer & -\\
	%\hline
\end{tabular}

\end{center}
\end{titlepage}

	\clearpage
	
	\cleardoublepage
	\tableofcontents
	\clearpage
	\listof{codesnippet}{Table des extraits}
	\clearpage
	%\cleardoublepage
	\section*{Introduction}
Ceci est une brève introduction.

Un serveur de test est mis à disposition à l'adresse suivante:

\href{http://parapheur.test.adullact.org}{http://parapheur.test.adullact.org}
	\clearpage
	%\cleardoublepage
	\section{Authentification}

\subsection{login}
\shadowbox{
\verb|/parapheur/api/login|
}\\

Cet appel ne nécessite aucune authentification (c'est d'ailleurs le seul). Une fois le ticket extrait du retour il doit être passé dans la query String\footnote{Par exemple: alf\_ticket=TICKET\_6a04816348fe8d9aab05ec7a43cfed22bdcdedd2} pour toutes les autres requêtes.

\begin{codesnippet}
\inputminted[frame=single,linenos,fontsize=\footnotesize]{javascript}{extraits/login_in.js}
\caption{Login requête entrante}
\label{snip:login_in}
\end{codesnippet}


\subsection{logout}
\shadowbox{
\verb|/parapheur/api/logout|
}\\

\begin{codesnippet}
\inputminted[frame=single,linenos,fontsize=\footnotesize]{javascript}{extraits/logout_in.js}
\caption{Logout requête entrante}
\label{snip:logout_in}
\end{codesnippet}

\begin{codesnippet}
\inputminted[frame=single,linenos,fontsize=\footnotesize]{javascript}{extraits/logout_out.js}
\caption{Logout retour}
\label{snip:logout_out}
\end{codesnippet}


\subsection{Validation du ticket d'authentification}
\shadowbox{
\verb|/api/login/{ticket}|
}\\

\clearpage
\section{Bureaux}

\subsection{getBureaux}
\shadowbox{
\verb|/parapheur/api/getBureaux|
}\\

Retourne la liste des bureaux dont l'utilisateur passé en paramètre est propriétaire ainsi que le nombre de dossier(s) dans présent(s) dans chaque corbeille (voir extrait~\ref{snip:getBureaux_out}) 

\begin{codesnippet}
\inputminted[frame=single,linenos,fontsize=\footnotesize]{javascript}{extraits/getBureaux_in.js}
\caption{getBureaux requête entrante}
\label{snip:getBureaux_in}
\end{codesnippet}

\subsection{getDossiersHeaders}
\shadowbox{
\verb|/parapheur/api/getDossiersHeaders|
}\\

\subsubsection{Entêtes de dossiers paginées}

Retourne une liste d'entêtes de dossiers issus de la bannette ``a-traiter''. (voir extraits~\ref{snip:getDossiersHeaders_in} et~\ref{snip:getDossiersHeaders_out}).

Lorsque l'on limite le nombre ($n$) de résultats \verb|getDossiersHeaders| retourne au plus $n + 1$ résultats afin d'indiquer une page suivante potentielle. Si \verb|pageSize| vaut 0 alors tout les résultats sont retournés.

\begin{codesnippet}
\inputminted[frame=single,linenos,fontsize=\footnotesize]{javascript}{extraits/getDossiersHeaders_in.js}
\caption{getDossiersHeaders requête entrante}
\label{snip:getDossiersHeaders_in}
\end{codesnippet}

\subsubsection{Filtres}
\label{Filtres}
Dans la requête des filtres il est possible d'utiliser les variables suivantes~\ref{table:filterable_fields} le type de la donnée est donné à titre indicatif car représenté sous forme de \verb|STRING| dans l'élément filtres de la requête.


\begin{figure}[H]
	\begin{center}
\begin{tabular}{c|c|c}
	\hline
	Nom & Type & Examples de valeurs \\
	\hline
	cm:name & STRING & re7 07\\
	ph:recuperable & BOOLEAN & false \\
	ph:reading-mandatory & BOOLEAN & false\\
	ph:signature-papier & BOOLEAN & false\\
	ph:tdtProtocole & STRING  & ACTES\\
	ph:dateLimite & DATE & null\\
	ph:soustypeMetier & STRING & Arrete du personel \\
	ph:confidentiel & BOOLEAN & false \\
	ph:termine & BOOLEAN & false \\
	ph:tdtNom & STRING & $S^2LOW$ \\
	ph:typeSignature & STRING & CMS\\
	ph:digital-signature-mandatory & BOOLEAN & false \\
	ph:sigFormat & STRING & PKCS$\sharp$7/single \\
	cm:modified & DATE & 2012-21-05 \\
	ph:typeMetier & STRING & Actes \\
	cm:created & DATE & 2012-17-01
	
\end{tabular}
\end{center}
\caption{Filtres potentiels}
\label{table:filterable_fields}

\end{figure}

Par exemple si l'on veut tout les dossiers commençant par ``Test''\footnote{On notera l'utilisation du caractère joker *} et ayant pour type ``HELIOS'' et sousType ``depense''
\begin{codesnippet}
\begin{minted}[frame=single,linenos,fontsize=\footnotesize]{javascript}
	"filtres": {
		"cm:name" : "Test*",
		"ph:typeMetier" : "HELIOS",
		"ph:soustypeMetier" : "depense"
	}
\end{minted}
\caption{Exemple de filtre à prédicats multiples}
\end{codesnippet}


\subsubsection{Filtres sur les champs de type date}

Les champs de type date peuvent être filtrés à l'aide d'un intervalle de date\footnote{Il est fort probable que cela fonctionne avec d'autres types énumérés par exemple les entiers}.

\begin{codesnippet}
\begin{minted}[frame=single,linenos,fontsize=\footnotesize]{javascript}
	"filtres": {
		"cm:created" : "[2012-06-01 TO NOW]",
	}
	
	"filtres": {
		"cm:created" : "[NOW TO 2012-06-01]",
	}
	
	"filtres": {
		"cm:created" : "[2012-06-01 TO 2012-06-06]",
	}
\end{minted}
\caption{Exemple de filtre sur des champs de type date}
\end{codesnippet}

\subsubsection{Filtres Complexes}
Il est possible de faire des filtres complexes contenant des clauses ``OR'' et ``AND''.

\begin{codesnippet}
\inputminted[frame=single,linenos,fontsize=\footnotesize]{javascript}{extraits/getDossiersHeaders_search_complex_in.js}
\caption{getDossiersHeaders Complex filters}
\label{snip:getTypologie_in}
\end{codesnippet}

\subsubsection{Spécification du Parent virtuel}

Par défaut les filtres retournent des résultats sur les dossiers provenant de la corbeille ``a-traiter''. Il est désormais possible de
choisir la corbeille (y compris virtuelle). Le tableau suivant~\ref{table:parent_values} récapitule les différentes corbeilles possibles.

\begin{figure}
\begin{center}
\begin{tabular}{c|p{7cm}}
	\hline
	Corbeille & Déscription \\
	\hline
        \verb|en-preparation| & contient les dossiers qui n'ont pas été encore émis  \\
        \verb|a-traiter| & contient les dossier que l'utilisateur doit traiter (signer, viser, etc ...) \\
        \verb|a-archiver| & contient les dossiers qui sont en fin de circuit ou à une étape d'archivage \\
        \verb|retournes| & contient les dossiers rejetés \\
        \verb|en-cours| & contient les dossiers emis par le proprietaire du bureau et qui suivent un circuit \\
        \verb|a-venir| & contient les dossiers qui vont arriver dans le bureau \\
        \verb|recuperables| & contient les dossiers récupérables (ie que l'on vient de viser) \\
        \verb|secretariat| & contient les dossiers envoyés au secretariat \\
        \verb|en-retard| & contient les dossiers dont la date limite < date du jour (valable uniquement pour le bureau emetteur) \\
        \verb|a-imprimer| & contient les dossiers à imprimer par la secretaire


\end{tabular}
\end{center}
\caption{Description des valeurs possibles de PARENT}
\label{table:parent_values}
\end{figure}


\subsection{getCircuit}
\shadowbox{\verb|/parapheur/api/getCircuit|
}\\


Retourne le circuit associé au dossierRef passé en paramètre (voir extrait ~\ref{snip:getCircuit_out}) 

\begin{codesnippet}
\inputminted[frame=single,linenos,fontsize=\footnotesize]{javascript}{extraits/getCircuit_in.js}
\caption{getCircuit in}
\label{snip:getCircuit_in}
\end{codesnippet}


\subsection{getTypologie}
\shadowbox{\verb|/parapheur/api/getTypologie|
}\\

Retourne la typologie accessible au bureau passé en paramètre (voir extrait~\ref{snip:getTypologie_out}).

\begin{codesnippet}
\inputminted[frame=single,linenos,fontsize=\footnotesize]{javascript}{extraits/getTypologie_in.js}
\caption{getTypologie in}
\label{snip:getTypologie_in}
\end{codesnippet}

\clearpage
\section{Dossiers}

\subsection{getDossier}
\shadowbox{\verb|/parapheur/api/getDossier|
}\\


Le flag \verb|recuperable| indique si le dernier acteur qui a effectué une action de validation récupérable (visa) peut récupérer le dossier.

Le champ \verb|nomTdt| correspond au nom du tiers de télétransmission renseigné dans le sous type associé au dossier (FAST, S2LOW ou SRCI).

Le champ \verb|protocoleTdt| correspond au protocole du tiers de télétransmission sélectionné pour ce dossier (HELIOS ou ACTES).

Le champ \verb|type| correspond au type métier du dossier.

Le champ \verb|sousType| correspond au sous type métier du dossier.

Le champ \verb|lectureObligatoire| signifie que la lecture du document principal est obligatoire à l'étape de signature.

Le champ \verb|signatureNumeriqueObligatoire| signifie que la signature électronique est obligatoire (pas de signature "papier")

Le champ \verb|actions| donne les possibilités d'actions pour l'utilisateur

Le champ \verb|metadonnees| renseigne les métadonnées du circuit et leurs valeurs pour le dossier actuel.

La partie documents de la réponse à \verb|getDossier| contient un champ \verb|downloadUrl| (pour avoir le chemin complet on effectue la concaténation du préfixe et de l'url\footnote{Dans le cas présent ça donne: http://mon.parapheur.fr/alfresco/service/api/node/workspace/SpacesStore/1b4b7715-c053-4cea-bed7-5be34373d565/content?alf\_ticket=TICKET\_6a04816348fe8d9aab05ec7a43cfed22bdcdedd2}) un champ \verb|size| en octets qui représente la taille du document éventuellement un champ \verb|visuelPdfUrl| qui renvoie une version pdf du document (voir extrait~\ref{snip:getDossier_out}). Un dossier contient un document principal (le premier de la liste) et éventuellement des annexes.

\begin{codesnippet}
\inputminted[frame=single,linenos,fontsize=\footnotesize]{javascript}{extraits/getDossier_in.js}
\caption{getDossier in}
\label{snip:getDossier_out}
\end{codesnippet}


\subsection{getImages}
\shadowbox{\verb|/parapheur/api/getImages|
}\\

Permet de récupérer le visuel du document principal sous forme de liste d'images (voir extrait~\ref{snip:getImages_out}).

\begin{codesnippet}
\inputminted[frame=single,linenos,fontsize=\footnotesize]{javascript}{extraits/getImages_in.js}
\caption{getImages in}
\label{snip:getImages_in}
\end{codesnippet}


\subsection{getMetadonnees}
\shadowbox{\verb|/parapheur/api/getMetadonnees|
}\\

Permet de récupérer les métadonnées relatives aux type et sous-typé donnés (voir extrait~\ref{snip:getMetadonnees_out}).

\begin{codesnippet}
\inputminted[frame=single,linenos,fontsize=\footnotesize]{javascript}{extraits/getMetadonnees_in.js}
\caption{getMetadonnees in}
\label{snip:getMetadonnees_in}
\end{codesnippet}


\subsection{deleteNodes}
\shadowbox{\verb|/parapheur/api/deleteNodes|
}\\

Supprime les noeuds, fournis sous forme de liste.

\begin{codesnippet}
\inputminted[frame=single,linenos,fontsize=\footnotesize]{javascript}{extraits/deleteNodes_in.js}
\caption{deleteNodes in}
\label{snip:deleteNodes_in}
\end{codesnippet}


\subsection{Actions}

Toutes les actions retournent des codes d'erreur en cas d'échec. Le \verb|nom| du dossier suivi de \verb|success| en cas de réussite, ou d'un code d'erreur en cas d'échec.
Toutes les fonctions suivantes peuvent renvoyer l'erreur \verb|app.ajax.msg.autorisation| si l'utilisateur n'a pas l'autorisation de faire cette action.

\subsubsection{visa}
\shadowbox{\verb|/parapheur/api/visa|
}\\

Permet de viser/émettre un dossier. Voir ~\ref{table:possible_error} pour les codes d'erreurs retournés.

\begin{figure}
	\begin{center}
        \begin{tabular}{c|c|c}
	        \hline
	        Message retourné & Erreur \\
	        \hline
	        app.ajax.msg.nameerror & Un dossier du même nom existe déjà \\
	        app.ajax.msg.circuit & Le circuit n'est pas défini\\
	        app.ajax.msg.metadata & Toutes les métadonnées obligatoire ne sont pas renseignées\\
	        app.ajax.msg.type & Type/Sous-Type non défini(s)\\
	        app.ajax.msg.document & Aucun document n'est défini pour ce dossier\\
        \end{tabular}
    \end{center}
    \caption{Erreurs possibles}
    \label{table:possible_error}
\end{figure}

\begin{codesnippet}
\inputminted[frame=single,linenos,fontsize=\footnotesize]{javascript}{extraits/visa_in.js}
\caption{visa in}
\label{snip:visa_in}
\end{codesnippet}


\subsubsection{reject}
\shadowbox{\verb|/parapheur/api/reject|
}\\

Rejet d'un dossier.

\begin{codesnippet}
\inputminted[frame=single,linenos,fontsize=\footnotesize]{javascript}{extraits/reject_in.js}
\caption{reject in}
\label{snip:reject_in}
\end{codesnippet}


\subsubsection{raz}
\shadowbox{\verb|/parapheur/api/razDossier|
}\\

Remise à zéro d'un dossier rejeté.

\begin{codesnippet}
\inputminted[frame=single,linenos,fontsize=\footnotesize]{javascript}{extraits/razDossier_in.js}
\caption{razDossier in}
\label{snip:razDossier_in}
\end{codesnippet}


\subsubsection{remorse}
\shadowbox{\verb|/parapheur/api/remorseDossier|
}\\

Récupération d'un dossier émis.

\begin{codesnippet}
\inputminted[frame=single,linenos,fontsize=\footnotesize]{javascript}{extraits/remorseDossier_in.js}
\caption{remorseDossier in}
\label{snip:remorseDossier_in}
\end{codesnippet}

\subsubsection{archive}
\shadowbox{\verb|/parapheur/api/archive|
}\\

Archivage d'un dossier terminé.

\begin{codesnippet}
\inputminted[frame=single,linenos,fontsize=\footnotesize]{javascript}{extraits/archiveDossier_in.js}
\caption{archiveDossier in}
\label{snip:archiveDossier_in}
\end{codesnippet}

\subsection{Création}

\subsubsection{beginCreateDossier}
\shadowbox{\verb|/parapheur/api/createDossier|
}\\

Crée un dossier vide avec un nom temporaire sous la forme \verb|Sans Titre| suivi du premier chiffre disponible.

\begin{codesnippet}
\inputminted[frame=single,linenos,fontsize=\footnotesize]{javascript}{extraits/createDossier_in.js}
\caption{createDossier requête entrante}
\label{snip:getDossier_in}
\end{codesnippet}

\begin{codesnippet}
\inputminted[frame=single,linenos,fontsize=\footnotesize]{javascript}{extraits/createDossier_out.js}
\caption{createDossier requête sortante}
\label{snip:getDossier_out}
\end{codesnippet}


\subsubsection{setCircuit}
\shadowbox{\verb|/parapheur/api/setCircuit|
}\\

Défini un circuit pour le dossier donné.

\begin{codesnippet}
\inputminted[frame=single,linenos,fontsize=\footnotesize]{javascript}{extraits/setCircuit_in.js}
\caption{setCircuit requête entrante}
\label{snip:setCircuit_in}
\end{codesnippet}



\subsubsection{setDossierProperties}
\shadowbox{\verb|/parapheur/api/setDossierProperties|
}\\

Enregistre les propriétés sur le dossier donné.

\begin{codesnippet}
\inputminted[frame=single,linenos,fontsize=\footnotesize]{javascript}{extraits/setDossierProperties_in.js}
\caption{setDossierProperties requête entrante}
\label{snip:setDossierProperties_in}
\end{codesnippet}


\subsubsection{addDocument}
\shadowbox{\verb|/parapheur/api/addDocument|
}\\

Ajoute un document au dossier donné. Retourne \verb|success| suivi de la référence du document créé (voir extrait~\ref{snip:addDocument_visuel_in}). En cas d'erreur, retourne \verb|error| suivi du type d'erreur rencontré.

\begin{codesnippet}
\inputminted[frame=single,linenos,fontsize=\footnotesize]{javascript}{extraits/addDocument_visuel_out.js}
\caption{addDocument requête sortante}
\label{snip:addDocument_visuel_out}
\end{codesnippet}


\subsubsection{removeDocument}
\shadowbox{\verb|/parapheur/api/removeDocument|
}\\

Supprime un document. Retourne le résultat de l'opération.

\begin{codesnippet}
\inputminted[frame=single,linenos,fontsize=\footnotesize]{javascript}{extraits/removeDocument_in.js}
\caption{removeDocument requête entrante}
\label{snip:removeDocument_in}
\end{codesnippet}

\begin{codesnippet}
\inputminted[frame=single,linenos,fontsize=\footnotesize]{javascript}{extraits/removeDocument_out.js}
\caption{removeDocument requête sortante}
\label{snip:removeDocument_out}
\end{codesnippet}


\subsubsection{finalizeCreateDossier}
\shadowbox{\verb|/parapheur/api/finalizeCreateDossier|
}\\

Permet d'autoriser ou de refuser l'émission du dossier donné.

\begin{codesnippet}
\inputminted[frame=single,linenos,fontsize=\footnotesize]{javascript}{extraits/finalizeCreateDossier_in.js}
\caption{finalizeCreateDossier requête entrante}
\label{snip:finalizeCreateDossier_in}
\end{codesnippet}

\subsection{Archives}

\subsubsection{getArchives}
\shadowbox{\verb|/parapheur/api/getArchives|
}\\

Retourne une liste d'entêtes d'archives. (voir extraits~\ref{snip:getArchives_in} et~\ref{snip:getArchives_out}).

Lorsque l'on limite le nombre ($n$) de résultats \verb|getArchives| retourne au plus $n + 1$ résultats afin d'indiquer une page suivante potentielle. Si \verb|pageSize| vaut 0 alors tout les résultats sont retournés. Voir ~\ref{Filtres} pour les filtres, le fonctionnement général étant le même.

\begin{codesnippet}
\inputminted[frame=single,linenos,fontsize=\footnotesize]{javascript}{extraits/getArchives_in.js}
\caption{getArchives requête entrante}
\label{snip:getArchives_in}
\end{codesnippet}

\subsubsection{getProp}
\shadowbox{\verb|/parapheur/api/getProp|
}\\

Permet de télécharger la propriété de type content donnée.\\
Attention : Cette fonction s'appelle en \verb|GET|.\\

Exemple d'utilisation :\\
Récupération du document original d'une archive :\\
\verb|parapheur/api/getProp?dossier=workspace://SpacesStore/47d8ab48-3704-41b3-aed7-4ee1ee0a6119&property=ph:original|


\clearpage
\section{Annotations}

\subsection{addAnnotation}
\shadowbox{\verb|/parapheur/api/addAnnotation|
}\\

Ajoute une annotation au dossier donné. Cette annotation est attaché à la \verb|page| du dossier. Permet l'enregistrement multiple (voir extrait~\ref{snip:addAnnotation_in})
Retourne la liste des uuid des annotations enregistrées.

\begin{codesnippet}
\inputminted[frame=single,linenos,fontsize=\footnotesize]{javascript}{extraits/addAnnotation_out.js}
\caption{addAnnotation requête sortante}
\label{snip:addAnnotation_out}
\end{codesnippet}

\subsection{updateAnnotation}
\shadowbox{\verb|/parapheur/api/updateAnnotation|
}\\

Met à jour l'annotation avec l'uuid donné (voir extrait~\ref{snip:updateAnnotation_in})

\subsection{getAnnotations}
\shadowbox{\verb|/parapheur/api/getAnnotations|
}\\

Permet de récupérer les annotations du dossier donné. Retourne la liste des annotations par page (voir extrait~\ref{snip:getAnnotation_out}).

\begin{codesnippet}
\inputminted[frame=single,linenos,fontsize=\footnotesize]{javascript}{extraits/getAnnotations_in.js}
\caption{getAnnotations requête entrante}
\label{snip:getAnnotations_in}
\end{codesnippet}


	\clearpage
	\section{Dossiers}

\subsection{getDossier}
\shadowbox{\verb|/parapheur/api/getDossier|
}\\


Le flag \verb|recuperable| indique si le dernier acteur qui a effectué une action de validation récupérable (visa) peut récupérer le dossier.

Le champ \verb|nomTdt| correspond au nom du tiers de télétransmission renseigné dans le sous type associé au dossier (FAST, S2LOW ou SRCI).

Le champ \verb|protocoleTdt| correspond au protocole du tiers de télétransmission sélectionné pour ce dossier (HELIOS ou ACTES).

Le champ \verb|type| correspond au type métier du dossier.

Le champ \verb|sousType| correspond au sous type métier du dossier.

Le champ \verb|lectureObligatoire| signifie que la lecture du document principal est obligatoire à l'étape de signature.

Le champ \verb|signatureNumeriqueObligatoire| signifie que la signature électronique est obligatoire (pas de signature "papier")

Le champ \verb|actions| donne les possibilités d'actions pour l'utilisateur

Le champ \verb|metadonnees| renseigne les métadonnées du circuit et leurs valeurs pour le dossier actuel.

La partie documents de la réponse à \verb|getDossier| contient un champ \verb|downloadUrl| (pour avoir le chemin complet on effectue la concaténation du préfixe et de l'url\footnote{Dans le cas présent ça donne: http://mon.parapheur.fr/alfresco/service/api/node/workspace/SpacesStore/1b4b7715-c053-4cea-bed7-5be34373d565/content?alf\_ticket=TICKET\_6a04816348fe8d9aab05ec7a43cfed22bdcdedd2}) un champ \verb|size| en octets qui représente la taille du document éventuellement un champ \verb|visuelPdfUrl| qui renvoie une version pdf du document (voir extrait~\ref{snip:getDossier_out}). Un dossier contient un document principal (le premier de la liste) et éventuellement des annexes.

\begin{codesnippet}
\inputminted[frame=single,linenos,fontsize=\footnotesize]{javascript}{extraits/getDossier_in.js}
\caption{getDossier in}
\label{snip:getDossier_out}
\end{codesnippet}


\subsection{getImages}
\shadowbox{\verb|/parapheur/api/getImages|
}\\

Permet de récupérer le visuel du document principal sous forme de liste d'images (voir extrait~\ref{snip:getImages_out}).

\begin{codesnippet}
\inputminted[frame=single,linenos,fontsize=\footnotesize]{javascript}{extraits/getImages_in.js}
\caption{getImages in}
\label{snip:getImages_in}
\end{codesnippet}


\subsection{getMetadonnees}
\shadowbox{\verb|/parapheur/api/getMetadonnees|
}\\

Permet de récupérer les métadonnées relatives aux type et sous-typé donnés (voir extrait~\ref{snip:getMetadonnees_out}).

\begin{codesnippet}
\inputminted[frame=single,linenos,fontsize=\footnotesize]{javascript}{extraits/getMetadonnees_in.js}
\caption{getMetadonnees in}
\label{snip:getMetadonnees_in}
\end{codesnippet}


\subsection{deleteNodes}
\shadowbox{\verb|/parapheur/api/deleteNodes|
}\\

Supprime les noeuds, fournis sous forme de liste.

\begin{codesnippet}
\inputminted[frame=single,linenos,fontsize=\footnotesize]{javascript}{extraits/deleteNodes_in.js}
\caption{deleteNodes in}
\label{snip:deleteNodes_in}
\end{codesnippet}


\subsection{Actions}

Toutes les actions retournent des codes d'erreur en cas d'échec. Le \verb|nom| du dossier suivi de \verb|success| en cas de réussite, ou d'un code d'erreur en cas d'échec.
Toutes les fonctions suivantes peuvent renvoyer l'erreur \verb|app.ajax.msg.autorisation| si l'utilisateur n'a pas l'autorisation de faire cette action.

\subsubsection{visa}
\shadowbox{\verb|/parapheur/api/visa|
}\\

Permet de viser/émettre un dossier. Voir ~\ref{table:possible_error} pour les codes d'erreurs retournés.

\begin{figure}
	\begin{center}
        \begin{tabular}{c|c|c}
	        \hline
	        Message retourné & Erreur \\
	        \hline
	        app.ajax.msg.nameerror & Un dossier du même nom existe déjà \\
	        app.ajax.msg.circuit & Le circuit n'est pas défini\\
	        app.ajax.msg.metadata & Toutes les métadonnées obligatoire ne sont pas renseignées\\
	        app.ajax.msg.type & Type/Sous-Type non défini(s)\\
	        app.ajax.msg.document & Aucun document n'est défini pour ce dossier\\
        \end{tabular}
    \end{center}
    \caption{Erreurs possibles}
    \label{table:possible_error}
\end{figure}

\begin{codesnippet}
\inputminted[frame=single,linenos,fontsize=\footnotesize]{javascript}{extraits/visa_in.js}
\caption{visa in}
\label{snip:visa_in}
\end{codesnippet}


\subsubsection{reject}
\shadowbox{\verb|/parapheur/api/reject|
}\\

Rejet d'un dossier.

\begin{codesnippet}
\inputminted[frame=single,linenos,fontsize=\footnotesize]{javascript}{extraits/reject_in.js}
\caption{reject in}
\label{snip:reject_in}
\end{codesnippet}


\subsubsection{raz}
\shadowbox{\verb|/parapheur/api/razDossier|
}\\

Remise à zéro d'un dossier rejeté.

\begin{codesnippet}
\inputminted[frame=single,linenos,fontsize=\footnotesize]{javascript}{extraits/razDossier_in.js}
\caption{razDossier in}
\label{snip:razDossier_in}
\end{codesnippet}


\subsubsection{remorse}
\shadowbox{\verb|/parapheur/api/remorseDossier|
}\\

Récupération d'un dossier émis.

\begin{codesnippet}
\inputminted[frame=single,linenos,fontsize=\footnotesize]{javascript}{extraits/remorseDossier_in.js}
\caption{remorseDossier in}
\label{snip:remorseDossier_in}
\end{codesnippet}

\subsubsection{archive}
\shadowbox{\verb|/parapheur/api/archive|
}\\

Archivage d'un dossier terminé.

\begin{codesnippet}
\inputminted[frame=single,linenos,fontsize=\footnotesize]{javascript}{extraits/archiveDossier_in.js}
\caption{archiveDossier in}
\label{snip:archiveDossier_in}
\end{codesnippet}

\subsection{Création}

\subsubsection{beginCreateDossier}
\shadowbox{\verb|/parapheur/api/createDossier|
}\\

Crée un dossier vide avec un nom temporaire sous la forme \verb|Sans Titre| suivi du premier chiffre disponible.

\begin{codesnippet}
\inputminted[frame=single,linenos,fontsize=\footnotesize]{javascript}{extraits/createDossier_in.js}
\caption{createDossier requête entrante}
\label{snip:getDossier_in}
\end{codesnippet}

\begin{codesnippet}
\inputminted[frame=single,linenos,fontsize=\footnotesize]{javascript}{extraits/createDossier_out.js}
\caption{createDossier requête sortante}
\label{snip:getDossier_out}
\end{codesnippet}


\subsubsection{setCircuit}
\shadowbox{\verb|/parapheur/api/setCircuit|
}\\

Défini un circuit pour le dossier donné.

\begin{codesnippet}
\inputminted[frame=single,linenos,fontsize=\footnotesize]{javascript}{extraits/setCircuit_in.js}
\caption{setCircuit requête entrante}
\label{snip:setCircuit_in}
\end{codesnippet}



\subsubsection{setDossierProperties}
\shadowbox{\verb|/parapheur/api/setDossierProperties|
}\\

Enregistre les propriétés sur le dossier donné.

\begin{codesnippet}
\inputminted[frame=single,linenos,fontsize=\footnotesize]{javascript}{extraits/setDossierProperties_in.js}
\caption{setDossierProperties requête entrante}
\label{snip:setDossierProperties_in}
\end{codesnippet}


\subsubsection{addDocument}
\shadowbox{\verb|/parapheur/api/addDocument|
}\\

Ajoute un document au dossier donné. Retourne \verb|success| suivi de la référence du document créé (voir extrait~\ref{snip:addDocument_visuel_in}). En cas d'erreur, retourne \verb|error| suivi du type d'erreur rencontré.

\begin{codesnippet}
\inputminted[frame=single,linenos,fontsize=\footnotesize]{javascript}{extraits/addDocument_visuel_out.js}
\caption{addDocument requête sortante}
\label{snip:addDocument_visuel_out}
\end{codesnippet}


\subsubsection{removeDocument}
\shadowbox{\verb|/parapheur/api/removeDocument|
}\\

Supprime un document. Retourne le résultat de l'opération.

\begin{codesnippet}
\inputminted[frame=single,linenos,fontsize=\footnotesize]{javascript}{extraits/removeDocument_in.js}
\caption{removeDocument requête entrante}
\label{snip:removeDocument_in}
\end{codesnippet}

\begin{codesnippet}
\inputminted[frame=single,linenos,fontsize=\footnotesize]{javascript}{extraits/removeDocument_out.js}
\caption{removeDocument requête sortante}
\label{snip:removeDocument_out}
\end{codesnippet}


\subsubsection{finalizeCreateDossier}
\shadowbox{\verb|/parapheur/api/finalizeCreateDossier|
}\\

Permet d'autoriser ou de refuser l'émission du dossier donné.

\begin{codesnippet}
\inputminted[frame=single,linenos,fontsize=\footnotesize]{javascript}{extraits/finalizeCreateDossier_in.js}
\caption{finalizeCreateDossier requête entrante}
\label{snip:finalizeCreateDossier_in}
\end{codesnippet}

\subsection{Archives}

\subsubsection{getArchives}
\shadowbox{\verb|/parapheur/api/getArchives|
}\\

Retourne une liste d'entêtes d'archives. (voir extraits~\ref{snip:getArchives_in} et~\ref{snip:getArchives_out}).

Lorsque l'on limite le nombre ($n$) de résultats \verb|getArchives| retourne au plus $n + 1$ résultats afin d'indiquer une page suivante potentielle. Si \verb|pageSize| vaut 0 alors tout les résultats sont retournés. Voir ~\ref{Filtres} pour les filtres, le fonctionnement général étant le même.

\begin{codesnippet}
\inputminted[frame=single,linenos,fontsize=\footnotesize]{javascript}{extraits/getArchives_in.js}
\caption{getArchives requête entrante}
\label{snip:getArchives_in}
\end{codesnippet}

\subsubsection{getProp}
\shadowbox{\verb|/parapheur/api/getProp|
}\\

Permet de télécharger la propriété de type content donnée.\\
Attention : Cette fonction s'appelle en \verb|GET|.\\

Exemple d'utilisation :\\
Récupération du document original d'une archive :\\
\verb|parapheur/api/getProp?dossier=workspace://SpacesStore/47d8ab48-3704-41b3-aed7-4ee1ee0a6119&property=ph:original|


	\clearpage
	\section{Annotations}

\subsection{addAnnotation}
\shadowbox{\verb|/parapheur/api/addAnnotation|
}\\

Ajoute une annotation au dossier donné. Cette annotation est attaché à la \verb|page| du dossier. Permet l'enregistrement multiple (voir extrait~\ref{snip:addAnnotation_in})
Retourne la liste des uuid des annotations enregistrées.

\begin{codesnippet}
\inputminted[frame=single,linenos,fontsize=\footnotesize]{javascript}{extraits/addAnnotation_out.js}
\caption{addAnnotation requête sortante}
\label{snip:addAnnotation_out}
\end{codesnippet}

\subsection{updateAnnotation}
\shadowbox{\verb|/parapheur/api/updateAnnotation|
}\\

Met à jour l'annotation avec l'uuid donné (voir extrait~\ref{snip:updateAnnotation_in})

\subsection{getAnnotations}
\shadowbox{\verb|/parapheur/api/getAnnotations|
}\\

Permet de récupérer les annotations du dossier donné. Retourne la liste des annotations par page (voir extrait~\ref{snip:getAnnotation_out}).

\begin{codesnippet}
\inputminted[frame=single,linenos,fontsize=\footnotesize]{javascript}{extraits/getAnnotations_in.js}
\caption{getAnnotations requête entrante}
\label{snip:getAnnotations_in}
\end{codesnippet}

	\clearpage
	%\cleardoublepage
	\section{Annexes}

\begin{codesnippet}
\inputminted[frame=single,linenos,fontsize=\footnotesize]{javascript}{extraits/login_out.js}
\caption{Login retour}
\label{snip:login_out}
\end{codesnippet}

\begin{codesnippet}
\inputminted[frame=single,linenos,fontsize=\footnotesize]{javascript}{extraits/getCircuit_in.js}
\caption{getCircuit requête entrante}
\label{snip:getCircuit_in}
\end{codesnippet}

\begin{codesnippet}
\inputminted[frame=single,linenos,fontsize=\footnotesize]{javascript}{extraits/getCircuit_out.js}
\caption{getCircuit retour}
\label{snip:getCircuit_out}
\end{codesnippet}

\begin{codesnippet}
\inputminted[frame=single,linenos,fontsize=\footnotesize]{javascript}{extraits/getDossiersHeaders_out.js}
\caption{getDossiersHeaders retour}
\label{snip:getDossiersHeaders_out}
\end{codesnippet}

\begin{codesnippet}
\inputminted[frame=single,linenos,fontsize=\footnotesize]{javascript}{extraits/getDossiersHeaders_search_in.js}
\caption{getDossiersHeaders avec filtres requête entrante}
\label{snip:getDossiersHeaders_search_in}
\end{codesnippet}

\begin{codesnippet}
\inputminted[frame=single,linenos,fontsize=\footnotesize]{javascript}{extraits/getBureaux_out.js}
\caption{getBureaux retour}
\label{snip:getBureaux_out}
\end{codesnippet}


\begin{codesnippet}
\inputminted[frame=single,linenos,fontsize=\footnotesize]{javascript}{extraits/getDossiersHeaders_search_out.js}
\caption{getDossiersHeaders retour filtres}
\label{snip:getDossiersHeaders_search_out}
\end{codesnippet}

\begin{codesnippet}
\inputminted[frame=single,linenos,fontsize=\footnotesize]{javascript}{extraits/getTypologie_out.js}
\caption{geTypologie out}
\label{snip:getTypologie_out}
\end{codesnippet}

\begin{codesnippet}
\inputminted[frame=single,linenos,fontsize=\footnotesize]{javascript}{extraits/getDossier_out.js}
\caption{getDossier retour}
\label{snip:getDossier_out}
\end{codesnippet}


\end{document}
